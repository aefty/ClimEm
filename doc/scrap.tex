
\newpage







\newpage
We will discuss these penalty functions in further detail in the following paragraph.
%
The benchmark used in the fitting procedure is the multimodal mean benchmark of the various decay trajectories, which is referred to as the $\mu$-benchmark; see, Figure~\ref{fig:1} for visualization.
%
Let $\bb{y}^{\mu} \in \mathbb{R}^{T}$ denote the atmospheric decay trajectory for the $\mu$-benchmark for $T$ years after the introduction of the $100$ GtC pulse.
%
 Given the non-negative tuning coefficients $\rho_1$, $\rho_2$, and $\rho_3$, each which correspond to the respective penalty function, the proposed estimation procedure solves the following optimization probelm
 %
 \begin{align}\label{eq:obj}
 	\minimize{\{ \bb{a} , \bbt{m} \}}  \Bigg\{  
 	      \frac{1}{T} \Big \| \bb{M}[\ameq]_A - \bb{y}^\mu \Big \|_2+
 	 	 \rho_1 q_1(\ameq) +  
 		 \rho_1 q_2(\ameq) +   
 		 \rho_3 q_3(\ameq)  
 		 \Bigg\}.
 \end{align}
%
Note that the model fit error with respect to the $\mu$-benchmark, appearing as the first term in the above objective function, relates solely to atmospheric masses; the masses of other reservoirs are controlled through the penalty terms.
%
Typically, we aim to minimize~\eqref{eq:obj} so that we can accurately emulate the pulse event shortly after initiation; that is to say, for $T$ is not so large. 
%
In our tests we use $T=250$.
%
The dynamics further into the future carry much higher uncertainty and are less impactful to current decision-making policies and significance (see, e.g., Section~\ref{sec:econ} for further details).
%
%With regard to an aecomic anaylis the he economic model simulation discussed in Section~\ref{eq:X}.

%While the atmospheric fit errors mentioned earlier are more concerned shorter time-scales, the dynamics governing the absorption rates and sizes of non-atmospheric carbon reservoirs in the particular model are closely related to longer time scales.
%
Multiple dynamics govern the carbon-cycle, each operating on different time scales.
%
Typically, CO2 exchange within the atmosphere takes place over time scales spanning several years.
%
In contrast, carbon transfer with the land biosphere and the surface ocean can occur over periods ranging from decades to centuries. 
%
Meanwhile, carbon in deeper soils and the deep ocean cycles over time frames extending from centuries to millennia~\cite{IPCC}.
%
We can emulate these carbon-cycle process dynamics by controlling the dynamic time-scales of the operator $\bb{A}$, or equivalently, the magnitude of its eigenvalues.
%
Specifically, we encourage larger dynamic time-scales by penalizing solutions with penalty function
%
\begin{align}\label{eq:q1}
	q_1(\ameq) := -\frac{1}{n} \tr{\bb{A}[\ameq]} = -\frac{1}{n}\sum_{i=1}^n \bb{\lambda}_i(\bb{A}[\ameq]).
\end{align}
%
For any admissible parameters $\bb{a}$ and $\bbt{m}$ discussed in Section~\ref{sec:2.1}, the defined operator $\bb{A}$, has non-positive eigenvalues, making $q_1$ strictly positive. 





\newpage
%Along side, the fit error in ~\ref{eq:loss}, we now now dicuss the penalty functions $q_1$, $q_2$ and $q_3$ which  observed carbon cycle attributes

 while slower dynamics such as CO2 in deep oceans taking centuries to millennia cycle, we utlize a peneatlify fuckint 
%
Typically we wish to minimize the error for the model parameters such that we can accuracy emulate the pulse event soon after





%
The proposed carbon-cycle model and the different configurations, is a simplified approximation of a considerably more complex system. 
Rather than attempting to model processes across the spectrum of timescales, our goal here is to model the salient dynamics that play a significant role in the shortest and medium timescales. 
%
By fitting the model errors 

To penalize large dynamic timescales we encourage increasing eigenvalues magnitudes of $\bb{A}^b$, which corresponds to the penalty function

The proposed carbon-cycle model, along with various configurations, is a simplified approximation of a considerably more complex system. 
%
Rather than attempting to model processes across the spectrum of timescales, our goal here is to model the salient responses that play a significant role in the shortest and medium timescales. 
%
To penalize large dynamic timescales we encourage increasing eigenvalues magnitudes of $\bb{A}^b$, which corresponds to the penalty function
%
\begin{align}\label{eq:clim_model.7}
	q_1(\ameq) := \frac{1}{n} \tr{\bb{A}[\ameq]} = \frac{1}{n}\sum_{i=1}^n \bb{\lambda}_i(\bb{A}[\ameq]).
\end{align}
%



The proposed carbon-cycle model, along with various configurations, is a simplified approximation of a considerably more complex system. 
Rather than attempting to model processes across the spectrum of timescales, our goal here is to model the salient responses that play a significant role in the shortest and medium timescales. 
To penalize large dynamic timescales we encourage increasing eigenvalues magnitudes of $\bb{A}$, which corresponds to the penalty function
%
\begin{align}\label{eq:clim_model.7}
	q_1(\bbt{m}) := \frac{1}{p}\|   (\bbt{m} - \bbt{m}^*) \oslash \bbt{m}^*  \|_2.
\end{align}
%
Since the eigenvalues of the operator are strictly real and non-positive, the corresponding penalty function is strictly non-positive. 
This approach is further motivated by the economic model (see Section X), in which long-term damages hold little significance due to the discounting factor.





Following the pulse event, it is expected that the total pulse flux will be equally distributed between the ocean and land reservoirs in the short term. 
This assumption is corroborated by~\cite{joos2013carbon}, where the simulation of a $100$ GtC atmospheric emission (during preindustrial times) leads to an approximately equal partition of $60$ GtC of cumulative emissions in the ocean and land carbon reservoirs within $40$ to $60$ years after the pulse. 
We incorporate this observation by limiting the cumulative flux discrepancy between the ocean and land reservoirs using the penalty function
\begin{align}\label{eq:clim_model.8}
	q_3(\ameq) := \left \| \frac{  
	\bb{M}[ \ameq ]_{\text{O}}^{t_e}
	}{
	\bb{M}[ \ameq]_{\text{L}}^{t_e}} - 1 \right \|_2,
\end{align}
%
where $\bb{M}^{t_e}_{\text{O}}[\bb{A}]$ is the total mass of all ocean reservoirs (and similarly, the superscript $\text{L}$ denotes all land biosphere reservoirs) at time $t_e$, and $\eta>0$ is ratio  


 $={40}{25}=1.6$ 
In our experiments, we set $t_e=50$.

 
Let $\bb{y}^b$ represent the benchmark dataset of atmospheric carbon mass for the decaying $100$ GtC pulse, spanning a total of $T$ years after the pulse. 
In this pulse scenario, the emissions are defined as $\bb{e}^1_\text{A}=100$ and zero for all other cases. 
Moreover, since we begin with preindustrial conditions, it follows that $\bb{m}^{0}=\bb{m}^{\text{eq}}$.
Given the non-negative tuning coefficients $\rho_1$, $\rho_2$, and $\rho_3$, which correspond to each penalty function, the proposed estimation procedure seeks optimize
 %
 \begin{align}\label{eq:clim_model.9}
 	\minimize{ \substack{\{ \bb{A}^{b} \\ b  } \}   \; \forall\; b \in \mathcal{B} }  \left\{ \; \sum_{v \in \mathcal{B}} 
 		 \frac{1}{T} \left \| \bb{M}_{\text{A}}[\bb{A}^b] - \bb{y}^b  \right\|_2  + 
 		 \rho_1 q_1(\bb{A}^b,\bb{A}^\mu) +  
 		 \rho_1 q_2(\bb{A}^b) +   
 		 \rho_3 q_3(\bb{A}^b)  
 		 \right\},
 \end{align}
 %
where $\bb{M}_{\text{A}}[\bb{A}^b]$ represents the simulated values of atmospheric carbon content for the time period ranging from $t=1$ to $T$.


[Say more about optimization method used ... and bounds of optimizer. The max percentage transfer of mass form one reservoir to the next is 15\%]




\newpage
 referred to as the $\mu^{+}$ and $\mu^{-}$ benchmark, respectively.
%
This estimation procedure is applied to both the $3$SR and $4$PR carbon-cycle configuration. 
%
We note that the fitting process can be applied to any of the various benchmarks shown in Figure~\ref{fig:1}.
 
 


%
These 


the multimodal mean benchmark (denoted as $\mu$ benchmark). 

 

%



%

%
We first begin with formalizing these physical principals, encoding them in in three district penalty functions $q_1$, $q_2$ and $q_3$. 
%
Subsequently, we outline the comprehensive optimization process for the estimation procedure.

[more text here]


The variouse benchmark datasets are shown Figure~\ref{fig:2} for details. 
%
We employ $k$ scenarios to estimate the operator parameters, for example for $k=3$ we can have the multi-modal mean and two standard deviation extrema, denoted as $\mathcal{B}:=\{\mu,\mu^+,\mu^-\}$, or a $k=1$ direct fit to the  multi-modal, in which case we have  $\mathcal{B}:=\{\mu\}$

The proposed carbon-cycle model, along with various configurations, is a simplified approximation of a considerably more complex system. 
Rather than attempting to model processes across the spectrum of timescales, our goal here is to model the salient responses that play a significant role in the shortest and medium timescales. 
To penalize large dynamic timescales we encourage increasing eigenvalues magnitudes of $\bb{A}^b$, which corresponds to the penalty function
%
\begin{align}\label{eq:clim_model.7}
	q_1(\bb{A}) := \frac{1}{n} \tr{\bb{A}} = \frac{1}{n}\sum_{i=1}^n \bb{\lambda}_i(\bb{A}).
\end{align}
%
Since the eigenvalues of the operator are strictly real and non-positive, the corresponding penalty function is strictly non-positive. 
This approach is further motivated by the economic model (see Section X), in which long-term damages hold little significance due to the discounting factor.

Given a benchmark $b \in \mathcal{B}$, we obtain varying estimates of $\bb{A}^{b}$ for each $b$. 
The operator estimate $\bb{A}^{b}$ should be stable in the sense that the estimated parameters across the benchmarks should have similar values.
To reduce the variability amongst the estimates $\bb{A}^{b}$, we penalize the difference between the parameter estimates with respect to $b=\mu$.
We encode this attribute in the penalty function 
%
\begin{align}\label{eq:clim_model.6}
	q_2(\bb{A}) :=  \frac{ \left \| \bb{A} \bbt{m}^{\text{eq}} \right \|_2 }{\| \bbt{m}^{\text{eq}} \|_2 } ,
\end{align}
%
This penalty function results in a non-separable dependency between the parameter estimates. 
This dependency plays a critical role in constraining the range of potential operators that best fit the decay of atmospheric pulse emissions (see, e.g., Section X for further details.)
 %


Following the pulse event, it is expected that the total pulse flux will be equally distributed between the ocean and land reservoirs in the short term. 
This assumption is corroborated by~\cite{joos2013carbon}, where the simulation of a $100$ GtC atmospheric emission (during preindustrial times) leads to an approximately equal partition of $60$ GtC of cumulative emissions in the ocean and land carbon reservoirs within $40$ to $60$ years after the pulse. 
We incorporate this observation by limiting the cumulative flux discrepancy between the ocean and land reservoirs using the penalty function
\begin{align}\label{eq:clim_model.8}
	q_3(\bb{A}) := \left \| \frac{  
	\bb{M}[ \bb{a},\bb{m}^{\text{eq}} ]_{\text{O}}^{t_e}
	}{
	\bb{M}[ \bb{a},\bb{m}^{\text{eq}} ]_{\text{L}}^{t_e}} - \frac{\bbt{m}}{}  \right \|_2,
\end{align}
%
where $\bb{M}^{t_e}_{\text{O}}[\bb{A}]$ is the total mass of all ocean reservoirs (and similarly, the superscript $\text{L}$ denotes all land biosphere reservoirs) at time $t_e$, and $\eta>0$ is ratio  


 $={40}{25}=1.6$ 
In our experiments, we set $t_e=50$.

 
Let $\bb{y}^b$ represent the benchmark dataset of atmospheric carbon mass for the decaying $100$ GtC pulse, spanning a total of $T$ years after the pulse. 
In this pulse scenario, the emissions are defined as $\bb{e}^1_\text{A}=100$ and zero for all other cases. 
Moreover, since we begin with preindustrial conditions, it follows that $\bb{m}^{0}=\bb{m}^{\text{eq}}$.
Given the non-negative tuning coefficients $\rho_1$, $\rho_2$, and $\rho_3$, which correspond to each penalty function, the proposed estimation procedure seeks optimize
 %
 \begin{align}\label{eq:clim_model.9}
 	\minimize{ \substack{\{ \bb{A}^{b} \\ b  } \}   \; \forall\; b \in \mathcal{B} }  \left\{ \; \sum_{v \in \mathcal{B}} 
 		 \frac{1}{T} \left \| \bb{M}_{\text{A}}[\bb{A}^b] - \bb{y}^b  \right\|_2  + 
 		 \rho_1 q_1(\bb{A}^b,\bb{A}^\mu) +  
 		 \rho_1 q_2(\bb{A}^b) +   
 		 \rho_3 q_3(\bb{A}^b)  
 		 \right\},
 \end{align}
 %
where $\bb{M}_{\text{A}}[\bb{A}^b]$ represents the simulated values of atmospheric carbon content for the time period ranging from $t=1$ to $T$.


[Say more about optimization method used ... and bounds of optimizer. The max percentage transfer of mass form one reservoir to the next is 15\%]

 


\subsection{Calibration}



\subsubsection{Direct Fit}


\begin{table}[t]
\centering
\small
\begin{tabular}{ccccccccc}
$$&
$\bb{m}_{\text{A}}^{{\text{eq}}}$&
$\bb{m}_{\text{O}_1}^{{\text{eq}}}$&
$\bb{m}_{\text{O}_2}^{{\text{eq}}}$&
$\bb{m}_{\text{L}_1}^{{\text{eq}}}$&
$\bb{A}_{\text{A}   \to \text{O}_1}$& 
$\bb{A}_{\text{O}_1 \to \text{O}_2}$& 
$\bb{A}_{\text{A}   \to \text{L}  }$&
$\bb{\tau}$
\\[4pt]
%
\toprule
\toprule
	\textbf{3SR} & 589 & 683 & 1,174 & -   & 0.08272 & 0.01465 & -       & [6,64]     \\
	\textbf{4PR} & 589 & 748 & 1,166 & 85  & 0.05777 & 0.01087 & 0.04148 & [3,10,84]
\end{tabular}
\label{tab:1}
\caption{Equilibrium masses in the different carbon reservoirs are shown for each configuration. For all configurations, $\text{O}_1$ and $\text{O}_2$ represent the upper and lower-ocean, respectively. In the $4$PR model, $\text{L}_1$ denotes the total land-biosphere equilibrium mass, while the $5$PR configuration subdivides the land-biosphere into vegetation $\text{L}_1$ and soils $\text{L}_2$. These masses correspond to the 1765 conditions, when the Earth's carbon cycle is assumed to be at equilibrium (for details, refer to \cite{2431585063dd4b78b890f885bb19642e} and the references therein).
    }
\label{tab:1}
\end{table}







Notice that, for any valid estimate of $\bb{A}^v$, adhering to the definition in \eqref{eq:clim_model.4}, the ratio of the long-term cumulative fluxes of the pulse into the ocean and land biosphere will asymptotically approach that of the ratios of $\bb{m}^{\text{eq}}$.
As such, in all model configuration, we will have over $90\%$ of pulse emissions in the ocean carbon reservoirs. 
With this said, when considering time windwos of hundres of years, the 















Let $\bb{y}$ denote the simulation dataset of atmospheric carbon mass of the decaying $100$ GtC pulse, and let $\bb{E}^{\text{pls}}$ be the corresponding emission. 
For this simulated scenario, the emissions are zero except for atmospheric emissions (denoted by the subscript ``A'') at time index $t=1$ denoted as $\mathbf{E}^{\text{pls}}_{\text{A},1}=100$. 
The loss function for the pulse simulation is than defined as 
%
\begin{align} \label{eq:clim_model.5}
	\mathcal{L}(\bb{A}|\bb{y}^{\text{pls}},\bb{E}^{\text{pls}}) 
	:= \Biggl( 
	\sum_{t}^T \Bigl(  
	\Bigl[ ( \bb{A} + \bb{I})\bb{M}_{:t-1}+\bb{E}^{\text{pls}}_{:t}\Bigr]_{\text{A}}-\bb{y}^{\text{pls}}_{t} 
	\Bigr)^{2} 
	\Biggr)^{\frac{1}{2}}
\end{align}
%
where the definition of $\bb{A}$ only depends on its strictly lower-triangular non-zero values, and $\bb{M}_{0:}=\bb{m}^{\text{eq}}$. 


(1) Reduce 




\begin{align}
	\underset{  \bb{A}^{\mu^+},\bb{A}^{\mu},\bb{A}^{\mu^-}}{\text{min}} \left\{ \sum_{ v \in \{\mu^+,\mu,\mu^-\} }    \mathcal{L}(\bb{A}^v|\bb{y}^{v},\bb{E}^{\text{pls}})  - \frac{1}{p} \tr{\bb{A}^{v}} + \| \bb{A}^{v} - \bb{A}^{\mu}  \|_{F}  + \mathcal{S}(\bb{A}^{v})^{} \right \}
\end{align}

















\newpage
Given the carbon emissions time-series $\bb{E} \in \mathbb{R}^{p \times T}$, with $\bb{E}_{:t}=\bb{e}^{(t)}$ being the the emission introduced at time $t$, and initial conditions $\bb{m}^{0}$, the simulated carbon-cycle time-series is than 
%




\begin{align}
	\mathcal{L}(\bb{a},\bb{m}^{(\text{eq})}|\bb{E}) :=  \left\|    \bb{M}(\bb{a},\bb{m}^{(\text{eq})}|\bb{E}) _{\text{A}:}   - \bb{y}    \right\|_2
\end{align}


\begin{align}
	\mathcal{L}(\bb{a},\bb{m}^{(\text{eq})}|\bb{E}) :=  \left ( \sum_{t=1}^{T}  \left( \left[  \left( \bb{A}(\bb{a},\bb{m}^{(\text{eq})} )+\bb{I} \right) \bb{m}^{(t-1)} + \bb{E}_{:t} \right]_{\text{A}}  - \bb{y}_{t} \right)^{2} \right)^{\frac{1}{2}}
\end{align}








\newpage

We utilize a test of s
Following the approche of X, we aim to fit the nunkonanvesl such that the atompohwer decay aptter of the simuatle carbon cycle corredpond ot the pulsed atsets. 
This howvever, is high probalimatic, as we only optimize with  $\bb{m}^{t}_{A}$

As described in the following section, we estimate the operator $\bb{A}$ by fitting our system response to a set of simulations conducted using various EMIS and EMICS models for a 100 GtC pulse in pre-industrial times. The fitting process is performed using one degree of freedom in the model \eqref{eq:model.1}, which corresponds to the carbon mass in the atmosphere. This makes the model highly underdetermined. 










Given the carbon emissions time-series $\bb{E} \in \mathbb{R}^{p \times T}$, with $\bb{E}_{:t}$ being the the emission introduced at time $t$, and initial conditions $\bb{m}^{0}$, the simulated carbon-cycle time-series is than 
%
\begin{align}
	\mathcal{S}(\bb{a},\bb{m}^{(\text{eq})} \; | \; \bb{E}):= \bb{S} \; \text{where}\;  \bb{S}_{:t} = \big(\bb{A}(\bb{a},\bb{m}^{(\text{eq})} )+\bb{I} \big) \bb{m}^{(t-1)} + \bb{E}_{:t} ,
\end{align}
%
is the the carbon masses in each reservoirs, with $\bb{S}_{:t}=\bb{m}^{(0)}$ (see, e.g.,~\eqref{eq:clim_model.1} for further details). 
Following the approach of X, we utilize a set of standardized tests derived from a scenario in which a 100 GtC pulse of CO2 is introduced into the atmosphere during preindustrial times, with the carbon cycle assumed to be at equilibrium~\cite{joos2013carbon}. 
This scenario corresponds to the $\bb{m}^{(0)}=\bb{m}^{(\text{eq})}$ and $\bb{E}_{0,0}=100$ and $\bb{E}_{i,t}=0$ and zero everywhere else.  
Figure~\ref{fig:2} showcases this dataset for the various Earth System Models.
If we consider fitting the $\bb{a}$ and $\bb{m}^{(\text{eq})}$
\begin{align}
	\mathcal{L}_{\bb{E}} (\bb{a},\bb{m}^{(\text{eq})}) = \| \mathcal{S}^{AT}(\bb{a},\bb{m}^{(\text{eq})} | \bb{E}) - \bb{y} \|
\end{align}



The calibration process requres the fitting of the operator $\bb{A}$ such that simulation with the 100 Gtc Pulse corresponds tot eh  












\newpage

The unknown values needed to uniquely define $\bb{A}$ are the equilibrium masses $\bb{m}^{\text{eq}}$ and the strictly lower-triangular nonzero entries $\bb{A}_{ij}$ for all $i<j$. The remaining elements, which comprise of the upper-triangular nonzero entries of $\bb{A}$ (denoted as the black entries in the matrix sketch in Figure~\ref{fig:1}), can be determined sequentially as
%

%
 
 



To estimate the operator values, we will utilize the controlled simulations conducted by [JOOS], wherein a 100 GtC carbon pulse is induced in pre-industrial time at equilibrium. 

The simulation datasets comprise of a multi-model analysis of various Earth System models of varying complexity. 


 includgion variouse, model s simulations will be allowed to evolve over different time spans. The tests will include EMIS and EMICS (Earth System Models of Intermediate Complexity).



Earth system Models of Intermediate Complexity (EMICs) provide simulations of millennial timescale climate change, and are
used as tools to interpret and expand upon the results of more
comprehensive models



\newpage
As described in the following section, we estimate the operator $\bb{A}$ by fitting our system response to a set of simulations conducted using various EMIS and EMICS models for a 100 GtC pulse in pre-industrial times. The fitting process is performed using one degree of freedom in the model \eqref{eq:model.1}, which corresponds to the carbon mass in the atmosphere. This makes the model highly underdetermined. As such, different to what as done in CDICE an DICE, we assume $\bb{m}^{\text{eq}}$ to be given that of the libtratue, in partiuclar we assume  

%%%%%%%%%%%%%%%%%%%%%%%%%%%%%%%%%%%%%%%%%%%%%%%
\newpage
\subsection{Optimization} 
%%%%%%%%%%%%%%%%%%%%%%%%%%%%%%%%%%%%%%%%%%%%%%%
Define $\mathbf{x}$ as the model's unknown parameters, including the concatenation of the lower triangular non-zero entries of $\mathbf{A}$, represented by $\mathbf{a}$, and the equilibrium masses $\mathbf{m}^{\text{eq}}$. Then for $\mathbf{x}=[\mathbf{a},\mathbf{m}^{\text{eq}}]$, we define a forward simulation of the carbon 

\begin{align}
	\mathcal{S}(\bb{a},\bb{m}^{\text{eq}},t) := (\bb{A}(\bb{a})  + \bb{I})\bb{m}_0
\end{align}




 $\mathbf{A}$ which are the non-zero entries in the strictly lower triangular part of $\mathbf{A}$, and $\mathbf{b}$.



$\bb{x}$ denote the unkown paramters for the model, for ezample, for $4SR$ model, we have  $x\bb{x}=[\bb{A}_{21}$






 
\begin{align}
    \argmin{ \bb{A} \in \mathcal{C} }{\left\{ 
    \bigg\| \mathcal{S}_{\text{AT}}(\bb{A},T) - \bb{y}  \bigg \| + \alpha \| \bb{A} - \bb{A}^\top + \text{Diag}(\bb{A})\|_{F} + \eta \sum_{k} \| \mathcal{R}^{k}(\bb{A})
    \|_2 \right\}
    }
\end{align}


\begin{align}
    \argmin{ \bb{m}^{\text{eq}}, \bb{a}^{\ekk} \forall \ekk }{\left\{ \sum_{\ekk} \left[ \frac{1}{T} \left\| \bb{m}(\bb{a}^{\ekk},\bb{m}^{\text{eq}},T)_{\text{AT}} - \bb{b}^{\ekk}(T) \right\|_{2} 
    + \alpha g(\bb{a}^{\ekk},\bb{m}^{\text{eq}}) 
    + \eta   h(\bb{a}^{\ekk},\bb{m}^{\text{eq}}) 
    \right ] \right\}
    }
\end{align}
$\bb{A}_{ij}$ is is the fraction of mass that is transferred from reservoir $j$ to $i$.






\newpage

The system in \eqref{eq:model.1} is stable for $h=1$, with $|1+h\bb{\lambda}_i| \leqslant 1$ for all $i$. This stability allows for the use of a straightforward explicit Euler time-stepping scheme as a solution method. 


Given any two es

For a fixed $\bb{m}^{\text{eq}}$, we can see that any two model estimates 

For any two estimates of the operator $A$ the linear interpolant between two variables is $A$ and $B$ 



Form the linearity of the operator and assuming a fixed value for $\bb{m}^{\text{eq}}$ ), it follows that an two estimates of $\bb{A}$ for a given model configuration can be scaled between 

Given different estimates of the operator, for example the mean and two extremes, denoted here as $\bbh{A}^\text{m}$ and $\bbh{A}^\text{a}$ and $\bbh{A}^\text{a}$, respecitivly.







for example mean, and two extrea denoted as $\bbh{A}^a$ and $\bbh{A}^b$, with the same estimate equilbrium value $\bbh{m}^{\text{eq}}$, we can inteproaltino between operations with t
%
it follows directly that we can construct a superposition of the two operators. In estimating the unknowns of the carbon-cycle model, we may wish to capture not just the mean dynamics but also the extremes. 
%
\begin{align}
	\bbh{A}^{\alpha} = 
	\begin{dcases*}
       (1-\alpha)\bbh{A}^{\text{m}} +  \alpha \bbh{A}^{\text{a}} , & if $ 1 > \alpha \geqslant 0 $,\\
       (1+\alpha)\bbh{A}^{\text{m}}  - \alpha \bbh{A}^{\text{b}} , & if $ -1 < \alpha  < 0 $.
       \end{dcases*}
\end{align}
Given a pair of operator estimates of the same model  and which share the same  equilibrium condition $ \bbb{m}$, we can construct a weighted sum of the models  














In modeling carbon-cycle model, we may wish to capture a range of dynmaics which can span two extreme dynamics but also everything in between. The linearity of the model, provides a great deal of simplicity for this.  Consider three estimates of the operator $\bbh{A}^{m}$ for the mean and $\bbh{A}^{a}$ and $\bbh{A}^{b}$ as the two extremes of the dynamics of the carbon-cycle. 








In estimating the unknowns of the carbon-cycle model, we may wish to capture not just the median dynamics but also the extreams.  


 



Due to the lineary of the formulation, we can levelarge the princap of superpostion to attaina  of the model, we can  of models of the same type which share the same equalibrum condition $\bbb{m}$, we can construction a superposition of a model $\bb{A}$











 Notice that the numbering assigned to the reservoirs represents their sequential ordering and carries no physical interpretation; for example, $\text{O}_1$ and $\text{O}_2$ do not correspond to upper or middle ocean layers.




%In the section to follow we outline estimaiton procieesr for the the unkonwns in the operators, that is the vlues of $\bb{A}$ and $\bb{}$








%
Given a pair of operator estimates of the same model  and which share the same  equilibrium condition $ \bbb{m}$, we can construct a weighted sum of the models  
Given a pair of operator estimates of the same model  and which share the same  equilibrium condition $ \bbb{m}$, we can construct a weighted sum of the models  
Given a pair of operator estimates of the same model  and which share the same  equilibrium condition $ \bbb{m}$, we can construct a weighted sum of the models  
Given a pair of operator estimates of the same model  and which share the same  equilibrium condition $ \bbb{m}$, we can construct a weighted sum of the models  








%%%%%%%%%%%%%%%%%%%%%%%%%%%%%%%%%%%%%%%%%%%%%%%
\newpage
\subsection{Calibration}
%%%%%%%%%%%%%%%%%%%%%%%%%%%%%%%%%%%%%%%%%%%%%%%
Here is some text

\begin{table}[ht]
\centering
\small
\begin{tabular}{ccccccc}
\multirow{2}{*}{Configuration} & 
\multicolumn{5}{c}{Mass Flow Coefficients ($\num{1e-2}$)  \vspace{0.2em}}\\   &
$\bb{A}_{21} / \footnotesize{\too{$\text{AT}$}{$\text{O}_1$}}$&
$\bb{A}_{32} / \footnotesize{\too{$\text{O}_1$}{$\text{O}_2$}}$&
$\bb{A}_{43} / \footnotesize{\too{$\text{O}_2$}{$\text{O}_3$}}$&
$\bb{A}_{51} / \footnotesize{\too{$\text{AT}$}{$\text{L}_1$}}$& 
$\bb{A}_{65} / \footnotesize{\too{$\text{L}_1$}{$\text{L}_2$}}$\\ 
%
\toprule
\toprule
    \textbf{3SR}& & & & & \\
    \midrule
    MESMO & 
    0.730 & 0.031 & - & - & - \\
    MMM & 
    2.333 & 0.092 & - & - & - \\
    LOVECLIM & 
    4.553 & 0.116 & - & - & - \\
%
    \toprule
    \textbf{4SR}& & & & & \\
    \midrule
    MESMO & 
    2.929 & 0.225  & 0.044 & - & -  \\
    MMM & 
    6.220 & 0.795  & 0.061 & - & -  \\
    LOVECLIM & 
    29.983 & 1.255 & 0.089  & - & -  \\
%
    \toprule
    \textbf{5PR}& & & & & \\
    \midrule
    MESMO & 
    0.514 & 0.122 & 1.874 & 0.429  & -  \\
    MMM & 
    4.537 & 0.326 & 2.991 & 1.114  & -  \\
    LOVECLIM & 
    10.487 & 0.611 & 0.192 & 1.315 & -  \\
%
    \toprule
    \textbf{6PR}& & & & & \\
    \midrule
    MESMO & 
    0.809 & 0.096 & 4.693 & 4.959 & 0.214 \\
    MMM & 
    1.39 & 0.206 & 5.173 & 5.951 & 1.568  \\
    LOVECLIM & 
    3.322 & 0.271 & 3.393 & 2.231 & 11.49 \\
    \end{tabular}
    \caption{The estimated entries of operator $\bb{A}$, representing mass flow coefficients, were calibrated using a $100$ GtC pulse from equilibrium conditions across four model configurations and three distinct benchmarks. For further details, please refer to Section X.}
\end{table}




\begin{table}[ht]
\centering
\small
\begin{tabular}{ccccccc}
\multirow{2}{*}{Configuration} & 
\multicolumn{5}{c}{Equilibrium Masses \vspace{0.2em}}\\   &
$\bbb{m}_2 / \footnotesize{\text{O1}} $&
$\bbb{m}_3 / \footnotesize{\text{O2}} $&
$\bbb{m}_4 / \footnotesize{\text{O3}} $&
$\bbb{m}_5 / \footnotesize{\text{L1}} $&
$\bbb{m}_6 / \footnotesize{\text{L2}} $&\\ 
%
\toprule
\toprule
    \textbf{3SR} & 
    984 & 37016 & - & - & - \\
    \textbf{4SR} & 
    474 & 0.092 & - & - & - \\
    \textbf{5PR} & 
    4.553 & 0.116 & - & - & - \\
    \textbf{6PR} & 
    4.553 & 0.116 & - & - & - \\
    \end{tabular}
    \caption{The estimated equilibrium masses $\bbb{m}$ represent GtC at preindustrial times, specifically fixed to 1765. The atmospheric equilibrium mass (AT) is set at $\bbb{m}_1=589$ for all tests. It is important to note that these equilibrium conditions are shared across all model configurations, irrespective of the chosen benchmark. For more information, please refer to Section X.}
\end{table}


\begin{table}[ht]
\centering
\begin{tabular}{ccccc}
         & \textbf{3SR} & \textbf{4SR} & \textbf{5PR} & \textbf{6PR} \\
         \toprule
MESMO    & 7.13 / 23.16   & 0.53/8.21  & 7.14/22.63 & 0.43/5.13 \\
MMM      & 3.78 / 14.27   & 1.66/5.42  & 2.91/11.29 & 1.57/5.01 \\
LOVECLIM & 3.76 / 7.74    & 0.35/2.92  & 2.73/10.37 & 0.74/3.53 \\
\midrule
Average  & 4.89/15.06 & 0.85/5.52 & 4.26/14.76 & 0.91/4.56 
\end{tabular}
    \caption{Average absolute and maximum error for an in sample test for $1,000$ years.}
\end{table}

\begin{table}[]
\begin{tabular}{ccccc}
         & \textbf{3SR} & \textbf{4SR} & \textbf{5PR} & \textbf{6PR} \\
         \toprule
MESMO    & [58.  5844.] & [  11.  253. 3975.] & [  52.  138. 7142.] & [   7.   24.  263. 3871.] \\
MMM      & [21.  1504.] & [   6.   64. 1832.] & [  16.   54. 1676.] & [   7.   11.   68. 1884.] \\
LOVECLIM & [2.   1191.] & [   5.   25. 1532.] & [   7.   44. 1312.] & [   4.    8.   32. 1574.]  \\
\end{tabular}
    \caption{The dynamic time scales $\bb{\tau}$ for each model across four model configurations and the three distinct benchmarks MESMO,MMM and LOVECLIM. Notice that there are $\bb{\tau}_i=\bb{\lambda}^{-1}_i$ for all $\bb{\lambda}^{-1}_i \neq 0$ and thus there are $p-1$ dyanmic time scales for each gevne model. }
\end{table}



