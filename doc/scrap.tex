
\newpage


Notice that, for any valid estimate of $\bb{A}^v$, adhering to the definition in \eqref{eq:clim_model.4}, the ratio of the long-term cumulative fluxes of the pulse into the ocean and land biosphere will asymptotically approach that of the ratios of $\bb{m}^{\text{eq}}$.
As such, in all model configuration, we will have over $90\%$ of pulse emissions in the ocean carbon reservoirs. 
With this said, when considering time windwos of hundres of years, the 















Let $\bb{y}$ denote the simulation dataset of atmospheric carbon mass of the decaying $100$ GtC pulse, and let $\bb{E}^{\text{pls}}$ be the corresponding emission. 
For this simulated scenario, the emissions are zero except for atmospheric emissions (denoted by the subscript ``A'') at time index $t=1$ denoted as $\mathbf{E}^{\text{pls}}_{\text{A},1}=100$. 
The loss function for the pulse simulation is than defined as 
%
\begin{align} \label{eq:clim_model.5}
	\mathcal{L}(\bb{A}|\bb{y}^{\text{pls}},\bb{E}^{\text{pls}}) 
	:= \Biggl( 
	\sum_{t}^T \Bigl(  
	\Bigl[ ( \bb{A} + \bb{I})\bb{M}_{:t-1}+\bb{E}^{\text{pls}}_{:t}\Bigr]_{\text{A}}-\bb{y}^{\text{pls}}_{t} 
	\Bigr)^{2} 
	\Biggr)^{\frac{1}{2}}
\end{align}
%
where the definition of $\bb{A}$ only depends on its strictly lower-triangular non-zero values, and $\bb{M}_{0:}=\bb{m}^{\text{eq}}$. 


(1) Reduce 




\begin{align}
	\underset{  \bb{A}^{\mu^+},\bb{A}^{\mu},\bb{A}^{\mu^-}}{\text{min}} \left\{ \sum_{ v \in \{\mu^+,\mu,\mu^-\} }    \mathcal{L}(\bb{A}^v|\bb{y}^{v},\bb{E}^{\text{pls}})  - \frac{1}{p} \tr{\bb{A}^{v}} + \| \bb{A}^{v} - \bb{A}^{\mu}  \|_{F}  + \mathcal{S}(\bb{A}^{v})^{} \right \}
\end{align}

















\newpage
Given the carbon emissions time-series $\bb{E} \in \mathbb{R}^{p \times T}$, with $\bb{E}_{:t}=\bb{e}^{(t)}$ being the the emission introduced at time $t$, and initial conditions $\bb{m}^{0}$, the simulated carbon-cycle time-series is than 
%




\begin{align}
	\mathcal{L}(\bb{a},\bb{m}^{(\text{eq})}|\bb{E}) :=  \left\|    \bb{M}(\bb{a},\bb{m}^{(\text{eq})}|\bb{E}) _{\text{A}:}   - \bb{y}    \right\|_2
\end{align}


\begin{align}
	\mathcal{L}(\bb{a},\bb{m}^{(\text{eq})}|\bb{E}) :=  \left ( \sum_{t=1}^{T}  \left( \left[  \left( \bb{A}(\bb{a},\bb{m}^{(\text{eq})} )+\bb{I} \right) \bb{m}^{(t-1)} + \bb{E}_{:t} \right]_{\text{A}}  - \bb{y}_{t} \right)^{2} \right)^{\frac{1}{2}}
\end{align}








\newpage

We utilize a test of s
Following the approche of X, we aim to fit the nunkonanvesl such that the atompohwer decay aptter of the simuatle carbon cycle corredpond ot the pulsed atsets. 
This howvever, is high probalimatic, as we only optimize with  $\bb{m}^{t}_{A}$

As described in the following section, we estimate the operator $\bb{A}$ by fitting our system response to a set of simulations conducted using various EMIS and EMICS models for a 100 GtC pulse in pre-industrial times. The fitting process is performed using one degree of freedom in the model \eqref{eq:model.1}, which corresponds to the carbon mass in the atmosphere. This makes the model highly underdetermined. 










Given the carbon emissions time-series $\bb{E} \in \mathbb{R}^{p \times T}$, with $\bb{E}_{:t}$ being the the emission introduced at time $t$, and initial conditions $\bb{m}^{0}$, the simulated carbon-cycle time-series is than 
%
\begin{align}
	\mathcal{S}(\bb{a},\bb{m}^{(\text{eq})} \; | \; \bb{E}):= \bb{S} \; \text{where}\;  \bb{S}_{:t} = \big(\bb{A}(\bb{a},\bb{m}^{(\text{eq})} )+\bb{I} \big) \bb{m}^{(t-1)} + \bb{E}_{:t} ,
\end{align}
%
is the the carbon masses in each reservoirs, with $\bb{S}_{:t}=\bb{m}^{(0)}$ (see, e.g.,~\eqref{eq:clim_model.1} for further details). 
Following the approach of X, we utilize a set of standardized tests derived from a scenario in which a 100 GtC pulse of CO2 is introduced into the atmosphere during preindustrial times, with the carbon cycle assumed to be at equilibrium~\cite{joos2013carbon}. 
This scenario corresponds to the $\bb{m}^{(0)}=\bb{m}^{(\text{eq})}$ and $\bb{E}_{0,0}=100$ and $\bb{E}_{i,t}=0$ and zero everywhere else.  
Figure~\ref{fig:2} showcases this dataset for the various Earth System Models.
If we consider fitting the $\bb{a}$ and $\bb{m}^{(\text{eq})}$
\begin{align}
	\mathcal{L}_{\bb{E}} (\bb{a},\bb{m}^{(\text{eq})}) = \| \mathcal{S}^{AT}(\bb{a},\bb{m}^{(\text{eq})} | \bb{E}) - \bb{y} \|
\end{align}



The calibration process requres the fitting of the operator $\bb{A}$ such that simulation with the 100 Gtc Pulse corresponds tot eh  












\newpage

The unknown values needed to uniquely define $\bb{A}$ are the equilibrium masses $\bb{m}^{\text{eq}}$ and the strictly lower-triangular nonzero entries $\bb{A}_{ij}$ for all $i<j$. The remaining elements, which comprise of the upper-triangular nonzero entries of $\bb{A}$ (denoted as the black entries in the matrix sketch in Figure~\ref{fig:1}), can be determined sequentially as
%

%
 
 



To estimate the operator values, we will utilize the controlled simulations conducted by [JOOS], wherein a 100 GtC carbon pulse is induced in pre-industrial time at equilibrium. 

The simulation datasets comprise of a multi-model analysis of various Earth System models of varying complexity. 


 includgion variouse, model s simulations will be allowed to evolve over different time spans. The tests will include EMIS and EMICS (Earth System Models of Intermediate Complexity).



Earth system Models of Intermediate Complexity (EMICs) provide simulations of millennial timescale climate change, and are
used as tools to interpret and expand upon the results of more
comprehensive models



\newpage
As described in the following section, we estimate the operator $\bb{A}$ by fitting our system response to a set of simulations conducted using various EMIS and EMICS models for a 100 GtC pulse in pre-industrial times. The fitting process is performed using one degree of freedom in the model \eqref{eq:model.1}, which corresponds to the carbon mass in the atmosphere. This makes the model highly underdetermined. As such, different to what as done in CDICE an DICE, we assume $\bb{m}^{\text{eq}}$ to be given that of the libtratue, in partiuclar we assume  

%%%%%%%%%%%%%%%%%%%%%%%%%%%%%%%%%%%%%%%%%%%%%%%
\newpage
\subsection{Optimization} 
%%%%%%%%%%%%%%%%%%%%%%%%%%%%%%%%%%%%%%%%%%%%%%%
Define $\mathbf{x}$ as the model's unknown parameters, including the concatenation of the lower triangular non-zero entries of $\mathbf{A}$, represented by $\mathbf{a}$, and the equilibrium masses $\mathbf{m}^{\text{eq}}$. Then for $\mathbf{x}=[\mathbf{a},\mathbf{m}^{\text{eq}}]$, we define a forward simulation of the carbon 

\begin{align}
	\mathcal{S}(\bb{a},\bb{m}^{\text{eq}},t) := (\bb{A}(\bb{a})  + \bb{I})\bb{m}_0
\end{align}




 $\mathbf{A}$ which are the non-zero entries in the strictly lower triangular part of $\mathbf{A}$, and $\mathbf{b}$.



$\bb{x}$ denote the unkown paramters for the model, for ezample, for $4SR$ model, we have  $x\bb{x}=[\bb{A}_{21}$






 
\begin{align}
    \argmin{ \bb{A} \in \mathcal{C} }{\left\{ 
    \bigg\| \mathcal{S}_{\text{AT}}(\bb{A},T) - \bb{y}  \bigg \| + \alpha \| \bb{A} - \bb{A}^\top + \text{Diag}(\bb{A})\|_{F} + \eta \sum_{k} \| \mathcal{R}^{k}(\bb{A})
    \|_2 \right\}
    }
\end{align}


\begin{align}
    \argmin{ \bb{m}^{\text{eq}}, \bb{a}^{\ekk} \forall \ekk }{\left\{ \sum_{\ekk} \left[ \frac{1}{T} \left\| \bb{m}(\bb{a}^{\ekk},\bb{m}^{\text{eq}},T)_{\text{AT}} - \bb{b}^{\ekk}(T) \right\|_{2} 
    + \alpha g(\bb{a}^{\ekk},\bb{m}^{\text{eq}}) 
    + \eta   h(\bb{a}^{\ekk},\bb{m}^{\text{eq}}) 
    \right ] \right\}
    }
\end{align}
$\bb{A}_{ij}$ is is the fraction of mass that is transferred from reservoir $j$ to $i$.






\newpage

The system in \eqref{eq:model.1} is stable for $h=1$, with $|1+h\bb{\lambda}_i| \leqslant 1$ for all $i$. This stability allows for the use of a straightforward explicit Euler time-stepping scheme as a solution method. 


Given any two es

For a fixed $\bb{m}^{\text{eq}}$, we can see that any two model estimates 

For any two estimates of the operator $A$ the linear interpolant between two variables is $A$ and $B$ 



Form the linearity of the operator and assuming a fixed value for $\bb{m}^{\text{eq}}$ ), it follows that an two estimates of $\bb{A}$ for a given model configuration can be scaled between 

Given different estimates of the operator, for example the mean and two extremes, denoted here as $\bbh{A}^\text{m}$ and $\bbh{A}^\text{a}$ and $\bbh{A}^\text{a}$, respecitivly.







for example mean, and two extrea denoted as $\bbh{A}^a$ and $\bbh{A}^b$, with the same estimate equilbrium value $\bbh{m}^{\text{eq}}$, we can inteproaltino between operations with t
%
it follows directly that we can construct a superposition of the two operators. In estimating the unknowns of the carbon-cycle model, we may wish to capture not just the mean dynamics but also the extremes. 
%
\begin{align}
	\bbh{A}^{\alpha} = 
	\begin{dcases*}
       (1-\alpha)\bbh{A}^{\text{m}} +  \alpha \bbh{A}^{\text{a}} , & if $ 1 > \alpha \geqslant 0 $,\\
       (1+\alpha)\bbh{A}^{\text{m}}  - \alpha \bbh{A}^{\text{b}} , & if $ -1 < \alpha  < 0 $.
       \end{dcases*}
\end{align}
Given a pair of operator estimates of the same model  and which share the same  equilibrium condition $ \bbb{m}$, we can construct a weighted sum of the models  














In modeling carbon-cycle model, we may wish to capture a range of dynmaics which can span two extreme dynamics but also everything in between. The linearity of the model, provides a great deal of simplicity for this.  Consider three estimates of the operator $\bbh{A}^{m}$ for the mean and $\bbh{A}^{a}$ and $\bbh{A}^{b}$ as the two extremes of the dynamics of the carbon-cycle. 








In estimating the unknowns of the carbon-cycle model, we may wish to capture not just the median dynamics but also the extreams.  


 



Due to the lineary of the formulation, we can levelarge the princap of superpostion to attaina  of the model, we can  of models of the same type which share the same equalibrum condition $\bbb{m}$, we can construction a superposition of a model $\bb{A}$











 Notice that the numbering assigned to the reservoirs represents their sequential ordering and carries no physical interpretation; for example, $\text{O}_1$ and $\text{O}_2$ do not correspond to upper or middle ocean layers.




%In the section to follow we outline estimaiton procieesr for the the unkonwns in the operators, that is the vlues of $\bb{A}$ and $\bb{}$








%
Given a pair of operator estimates of the same model  and which share the same  equilibrium condition $ \bbb{m}$, we can construct a weighted sum of the models  
Given a pair of operator estimates of the same model  and which share the same  equilibrium condition $ \bbb{m}$, we can construct a weighted sum of the models  
Given a pair of operator estimates of the same model  and which share the same  equilibrium condition $ \bbb{m}$, we can construct a weighted sum of the models  
Given a pair of operator estimates of the same model  and which share the same  equilibrium condition $ \bbb{m}$, we can construct a weighted sum of the models  








%%%%%%%%%%%%%%%%%%%%%%%%%%%%%%%%%%%%%%%%%%%%%%%
\newpage
\subsection{Calibration}
%%%%%%%%%%%%%%%%%%%%%%%%%%%%%%%%%%%%%%%%%%%%%%%
Here is some text

\begin{table}[ht]
\centering
\small
\begin{tabular}{ccccccc}
\multirow{2}{*}{Configuration} & 
\multicolumn{5}{c}{Mass Flow Coefficients ($\num{1e-2}$)  \vspace{0.2em}}\\   &
$\bb{A}_{21} / \footnotesize{\too{$\text{AT}$}{$\text{O}_1$}}$&
$\bb{A}_{32} / \footnotesize{\too{$\text{O}_1$}{$\text{O}_2$}}$&
$\bb{A}_{43} / \footnotesize{\too{$\text{O}_2$}{$\text{O}_3$}}$&
$\bb{A}_{51} / \footnotesize{\too{$\text{AT}$}{$\text{L}_1$}}$& 
$\bb{A}_{65} / \footnotesize{\too{$\text{L}_1$}{$\text{L}_2$}}$\\ 
%
\toprule
\toprule
    \textbf{3SR}& & & & & \\
    \midrule
    MESMO & 
    0.730 & 0.031 & - & - & - \\
    MMM & 
    2.333 & 0.092 & - & - & - \\
    LOVECLIM & 
    4.553 & 0.116 & - & - & - \\
%
    \toprule
    \textbf{4SR}& & & & & \\
    \midrule
    MESMO & 
    2.929 & 0.225  & 0.044 & - & -  \\
    MMM & 
    6.220 & 0.795  & 0.061 & - & -  \\
    LOVECLIM & 
    29.983 & 1.255 & 0.089  & - & -  \\
%
    \toprule
    \textbf{5PR}& & & & & \\
    \midrule
    MESMO & 
    0.514 & 0.122 & 1.874 & 0.429  & -  \\
    MMM & 
    4.537 & 0.326 & 2.991 & 1.114  & -  \\
    LOVECLIM & 
    10.487 & 0.611 & 0.192 & 1.315 & -  \\
%
    \toprule
    \textbf{6PR}& & & & & \\
    \midrule
    MESMO & 
    0.809 & 0.096 & 4.693 & 4.959 & 0.214 \\
    MMM & 
    1.39 & 0.206 & 5.173 & 5.951 & 1.568  \\
    LOVECLIM & 
    3.322 & 0.271 & 3.393 & 2.231 & 11.49 \\
    \end{tabular}
    \caption{The estimated entries of operator $\bb{A}$, representing mass flow coefficients, were calibrated using a $100$ GtC pulse from equilibrium conditions across four model configurations and three distinct benchmarks. For further details, please refer to Section X.}
\end{table}




\begin{table}[ht]
\centering
\small
\begin{tabular}{ccccccc}
\multirow{2}{*}{Configuration} & 
\multicolumn{5}{c}{Equilibrium Masses \vspace{0.2em}}\\   &
$\bbb{m}_2 / \footnotesize{\text{O1}} $&
$\bbb{m}_3 / \footnotesize{\text{O2}} $&
$\bbb{m}_4 / \footnotesize{\text{O3}} $&
$\bbb{m}_5 / \footnotesize{\text{L1}} $&
$\bbb{m}_6 / \footnotesize{\text{L2}} $&\\ 
%
\toprule
\toprule
    \textbf{3SR} & 
    984 & 37016 & - & - & - \\
    \textbf{4SR} & 
    474 & 0.092 & - & - & - \\
    \textbf{5PR} & 
    4.553 & 0.116 & - & - & - \\
    \textbf{6PR} & 
    4.553 & 0.116 & - & - & - \\
    \end{tabular}
    \caption{The estimated equilibrium masses $\bbb{m}$ represent GtC at preindustrial times, specifically fixed to 1765. The atmospheric equilibrium mass (AT) is set at $\bbb{m}_1=589$ for all tests. It is important to note that these equilibrium conditions are shared across all model configurations, irrespective of the chosen benchmark. For more information, please refer to Section X.}
\end{table}


\begin{table}[ht]
\centering
\begin{tabular}{ccccc}
         & \textbf{3SR} & \textbf{4SR} & \textbf{5PR} & \textbf{6PR} \\
         \toprule
MESMO    & 7.13 / 23.16   & 0.53/8.21  & 7.14/22.63 & 0.43/5.13 \\
MMM      & 3.78 / 14.27   & 1.66/5.42  & 2.91/11.29 & 1.57/5.01 \\
LOVECLIM & 3.76 / 7.74    & 0.35/2.92  & 2.73/10.37 & 0.74/3.53 \\
\midrule
Average  & 4.89/15.06 & 0.85/5.52 & 4.26/14.76 & 0.91/4.56 
\end{tabular}
    \caption{Average absolute and maximum error for an in sample test for $1,000$ years.}
\end{table}

\begin{table}[]
\begin{tabular}{ccccc}
         & \textbf{3SR} & \textbf{4SR} & \textbf{5PR} & \textbf{6PR} \\
         \toprule
MESMO    & [58.  5844.] & [  11.  253. 3975.] & [  52.  138. 7142.] & [   7.   24.  263. 3871.] \\
MMM      & [21.  1504.] & [   6.   64. 1832.] & [  16.   54. 1676.] & [   7.   11.   68. 1884.] \\
LOVECLIM & [2.   1191.] & [   5.   25. 1532.] & [   7.   44. 1312.] & [   4.    8.   32. 1574.]  \\
\end{tabular}
    \caption{The dynamic time scales $\bb{\tau}$ for each model across four model configurations and the three distinct benchmarks MESMO,MMM and LOVECLIM. Notice that there are $\bb{\tau}_i=\bb{\lambda}^{-1}_i$ for all $\bb{\lambda}^{-1}_i \neq 0$ and thus there are $p-1$ dyanmic time scales for each gevne model. }
\end{table}



